\documentclass[a4paper,12pt]{article}
\usepackage[utf8]{inputenc}
\usepackage[italian]{babel}
\usepackage{hyperref}
\usepackage{amsmath}
\usepackage{xcolor}
\usepackage{geometry}
\usepackage{setspace}
\usepackage{graphicx}
\usepackage{fancyhdr}
\geometry{a4paper, margin=1in}


% Definizione di colori personalizzati
\definecolor{titlecolor}{RGB}{13, 82, 149}  % Blu scuro
\definecolor{sectioncolor}{RGB}{28, 69, 136} % Blu scuro più chiaro 
\definecolor{textcolor}{RGB}{0, 0, 0}    

% Aumento dell'interlinea
\setstretch{1.5}

% Personalizzazione dell'intestazione
\pagestyle{fancy}
\fancyhf{} % Cancella intestazioni e piè di pagina predefiniti
\fancyhead[L]{} % Lascia vuota la parte sinistra dell'intestazione
\fancyhead[R]{\includegraphics[width=1cm]{LogoAGA.jpeg}} % Logo in alto a destra
\fancyfoot[R]{\thepage} % Aggiunge il numero di pagina al centro del piè di pagina

% Personalizzazione dello stile del titolo e delle sezioni
\title{\textcolor{titlecolor}{\Huge Casi d'Uso - AGA \\ \text{Formato Breve}}}
\author{}
\date{}

\begin{document}

\maketitle

\section*{\textcolor{sectioncolor}{Introduzione}}
\textcolor{textcolor}{

Casi d’uso per un sistema di gestione eventi pensato per semplificare e automatizzare i processi legati all'organizzazione, alla vendita dei biglietti, alla gestione degli accessi e dei servizi durante l'evento. \\ Il sistema è destinato agli organizzatori, al personale operativo e ai clienti.
I casi d'uso descrivono le interazioni principali tra gli utenti e il sistema, includendo la creazione dell'evento, la gestione dei biglietti, l'acquisto, la verifica degli accessi, la gestione delle scorte e la visualizzazione della timeline dei servizi.

}

\section{\textcolor{sectioncolor}{Creazione Evento}}
\textcolor{textcolor}{
Un organizzatore accede al sistema e seleziona l'opzione per creare un nuovo evento. Inserisce i dettagli richiesti, come nome, data, orari e luogo. Una volta confermati i dettagli, il sistema verifica che siano validi e che non esistano già eventi programmati nello stesso mese. Se tutto è corretto, l’evento viene salvato nel database e diventa disponibile per la gestione dei biglietti. In caso di errori nei dati, il sistema notifica l’organizzatore specificando i problemi riscontrati.
}

\section{\textcolor{sectioncolor}{Aggiunta Biglietti}}
\textcolor{textcolor}{
Un organizzatore seleziona un evento attivo nel sistema e accede alla sezione di gestione dei biglietti. Inserisce i dettagli richiesti per ogni biglietto, come nome, cognome e quantità. Una volta confermati i dati, il sistema verifica che le vendite per l’evento siano aperte. Se tutto è corretto, genera un codice identificativo univoco per ciascun biglietto e salva le informazioni nel database. In caso di eventi con vendite chiuse, il sistema impedisce l’operazione e informa l’organizzatore.
}

\section{\textcolor{sectioncolor}{Apertura/Chiusura Vendite Biglietti}}
\textcolor{textcolor}{
Un organizzatore seleziona un evento attivo e accede alla sezione dedicata alla gestione delle vendite. Modifica lo stato delle vendite, scegliendo tra aperto o chiuso. Il sistema aggiorna il database con il nuovo stato e, se necessario, invia notifiche agli utenti interessati. Se i dati dell’evento sono incompleti, il sistema segnala un errore. Il sistema chiude automaticamente le vendite due settimane prima della data dell’evento.
}

\section{\textcolor{sectioncolor}{Acquisto Biglietti}}
\textcolor{textcolor}{
Un cliente accede alla piattaforma, seleziona un evento con vendite aperte e specifica il numero di biglietti desiderati. Inserisce i dati personali e procede al pagamento. Il sistema verifica il pagamento, genera un biglietto unico con QR code e lo invia via email al cliente. Se il pagamento non va a buon fine, l’operazione viene annullata. L’organizzatore, se necessario, può intervenire per aggiungere, modificare o eliminare un biglietto.
}

\section{\textcolor{sectioncolor}{Verifica Accesso Evento}}
\textcolor{textcolor}{
Lo staff all'ingresso scansiona il QR code del cliente utilizzando il sistema. Il sistema verifica la validità del QR code. Se il codice è valido, registra la presenza del cliente nel database e invalida il QR code per evitare usi futuri. Se il codice non è valido, il sistema avvisa lo staff con un messaggio di errore. La verifica deve avvenire in tempi ragionevoli.
}

\section{\textcolor{sectioncolor}{Gestione Guardaroba}}
\textcolor{textcolor}{
Lo staff del guardaroba scansiona il QR code del cliente tramite il sistema. Il sistema verifica la validità del codice. Se valido, assegna un numero unico, lo registra nel database e lo mostra allo staff. Se il QR code è invalido, il sistema invia un messaggio di errore allo staff. La generazione del numero deve seguire un ordine incrementale.
}

\section{\textcolor{sectioncolor}{Monitoraggio Scorte}}
\textcolor{textcolor}{
Il gestore accede al modulo dedicato alla gestione delle scorte. Il sistema mostra in tempo reale le quantità disponibili di cibo, ghiaccio e alcol. Se necessario, il gestore aggiorna le quantità registrate. Se i dati delle scorte sono assenti, il sistema avvisa della mancanza con un messaggio di errore. Le informazioni devono essere visualizzate e aggiornate in tempo reale per garantire accuratezza.
}

\section{\textcolor{sectioncolor}{Visualizzazione Timeline Servizio}}
\textcolor{textcolor}{
Il cameriere accede all’interfaccia personale tramite il sistema. Visualizza una timeline chiara e leggibile che mostra le attività e gli orari del servizio impostati per l’evento attivo. Se mancano dati essenziali o le scorte non sono disponibili, il sistema avvisa tempestivamente il cameriere.
}

\end{document}
