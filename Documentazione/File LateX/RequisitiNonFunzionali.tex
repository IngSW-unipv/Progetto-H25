\documentclass[a4paper,12pt]{article}
\usepackage[utf8]{inputenc}
\usepackage[italian]{babel}
\usepackage{hyperref}
\usepackage{amsmath}
\usepackage{xcolor}
\usepackage{geometry}
\usepackage{setspace}
\usepackage{graphicx}
\usepackage{fancyhdr}
\geometry{a4paper, margin=1in}

% Definizione di colori personalizzati
\definecolor{titlecolor}{RGB}{13, 82, 149}  % Blu scuro
\definecolor{sectioncolor}{RGB}{28, 69, 136} % Blu scuro pi\`u chiaro 
\definecolor{textcolor}{RGB}{0, 0, 0}    % Grigio scuro

\setstretch{1.5}

\pagestyle{fancy}
\fancyhf{} % Cancella intestazioni e pi\`e di pagina predefiniti
\fancyhead[L]{} % Lascia vuota la parte sinistra dell'intestazione
\fancyhead[R]{\includegraphics[width=1cm]{LogoAGA.jpeg}} % Logo in alto a destra
\fancyfoot[R]{\thepage} % Aggiunge il numero di pagina al centro del pi\`e di pagina

% Personalizzazione dello stile del titolo e delle sezioni
\title{\textcolor{titlecolor}{\Huge Requisiti Non Funzionali \vspace{0.2cm}}}
\author{}
\date{}

% Impostazione della dimensione del font
\renewcommand{\normalsize}{\fontsize{13pt}{16pt}\selectfont}
\renewcommand{\large}{\fontsize{15pt}{18pt}\selectfont}
\renewcommand{\Large}{\fontsize{17pt}{20pt}\selectfont}
\renewcommand{\LARGE}{\fontsize{20pt}{24pt}\selectfont}
\renewcommand{\Huge}{\fontsize{24pt}{28pt}\selectfont}

\begin{document}

\maketitle

\section*{\textcolor{sectioncolor}{Requisiti Tecnici}}
\textcolor{textcolor}{
\begin{itemize}
    \item \textbf{RNF1:} Il sistema deve essere eseguibile su qualsiasi dispositivo dotato di Java Virtual Machine (JVM).
    \item \textbf{RNF2:} L'interfaccia deve essere responsiva e adattarsi a schermi di diverse dimensioni, inclusi dispositivi mobili e desktop.
    \item \textbf{RNF3:} Il sistema deve rispondere alle operazioni principali (ad esempio, autenticazione, acquisto, verifica biglietti) in tempi accettabili per l'utente.
    \item \textbf{RNF4:} Il sistema deve essere in grado di gestire numerosi accessi simultanei al minuto durante eventi di picco.
    \item \textbf{RNF5:} Le password devono essere salvate in formato criptato.
    \item \textbf{RNF6:} I pagamenti devono essere effettuati tramite protocolli sicuri.
    \item \textbf{RNF7:} I QR code devono essere generati in modo univoco per ogni biglietto per prevenire duplicati o usi impropri.
    \item \textbf{RNF8:} Il sistema deve essere scalabile per supportare eventi con fino a migliaia di partecipanti.
    \item \textbf{RNF9:} Il sistema deve supportare connessioni con un database MySQL.
    \item \textbf{RNF10:} La documentazione tecnica deve includere linee guida chiare per sviluppatori e amministratori.
    \item \textbf{RNF11:} Il sistema deve rispettare le normative GDPR per il trattamento dei dati personali.
    \item \textbf{RNF12:} Il trattamento dei dati personali deve avvenire solo con il consenso esplicito degli utenti.
\end{itemize}
}

\section*{\textcolor{sectioncolor}{Requisiti Organizzativi}}
\textcolor{textcolor}{
\begin{itemize}
    \item \textbf{RNF13:} Il sistema deve essere completato entro il 28/02/2025.
    \item \textbf{RNF14:} Il sistema deve essere implementato utilizzando i linguaggi di programmazione Java e SQL.
    \item \textbf{RNF15:} La documentazione del sistema deve essere redatta in italiano.
    \item \textbf{RNF16:} La manutenzione programmata deve essere comunicata con almeno 48 ore di anticipo e non deve coincidere con la data dell’evento.
\end{itemize}
}

\section*{\textcolor{sectioncolor}{Requisiti Esterni}}
\textcolor{textcolor}{
\begin{itemize}
    \item \textbf{RNF17:} Il sistema deve appoggiarsi a un database esterno MySQL.
    \item \textbf{RNF18:} L’organizzatore, al momento della registrazione, deve autorizzare il sistema a salvare i propri dati personali su un database esterno.
    \item \textbf{RNF19:}\textbf{ }Le password salvate devono essere criptate.
    \item \textbf{RNF20:} Il database deve essere centralizzato per garantire una gestione uniforme dei dati.
\end{itemize}
}

\section*{\textcolor{sectioncolor}{Ulteriori Requisiti}}
\textcolor{textcolor}{
\begin{itemize}
    \item \textbf{RNF21:} La registrazione degli organizzatori deve avvenire tramite l'inserimento di dati personali, email e password.
\end{itemize}
}

\end{document}
