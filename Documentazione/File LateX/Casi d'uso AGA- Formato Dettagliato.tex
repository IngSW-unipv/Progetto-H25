\documentclass[a4paper,12pt]{article}
\usepackage[utf8]{inputenc}
\usepackage[italian]{babel}
\usepackage{hyperref}
\usepackage{amsmath}
\usepackage{xcolor}
\usepackage{geometry}
\usepackage{setspace}
\usepackage{graphicx}
\usepackage{fancyhdr}
\geometry{a4paper, margin=1.5in}

% Definizione di colori personalizzati
\definecolor{titlecolor}{RGB}{13, 82, 149}  % Blu scuro
\definecolor{sectioncolor}{RGB}{28, 69, 136} % Blu scuro pi\`u chiaro 
\definecolor{textcolor}{RGB}{0, 0, 0}    % Grigio scuro


\setstretch{1.5}

% Impostazione della dimensione del font
\renewcommand{\normalsize}{\fontsize{13pt}{16pt}\selectfont}
\renewcommand{\large}{\fontsize{15pt}{18pt}\selectfont}
\renewcommand{\Large}{\fontsize{17pt}{20pt}\selectfont}
\renewcommand{\LARGE}{\fontsize{20pt}{24pt}\selectfont}
\renewcommand{\Huge}{\fontsize{24pt}{28pt}\selectfont}

% Personalizzazione dell'intestazione
\pagestyle{fancy}
\fancyhf{} % Cancella intestazioni e pi\`e di pagina predefiniti
\fancyhead[L]{} % Lascia vuota la parte sinistra dell'intestazione
\fancyhead[R]{\includegraphics[width=1cm]{LogoAGA.jpeg}} % Logo in alto a destra
\fancyfoot[R]{\thepage} % Aggiunge il numero di pagina al centro del pi\`e di pagina

% Personalizzazione dello stile del titolo e delle sezioni
\title{\textcolor{titlecolor}{\Huge Casi d'Uso - AGA \\ \text{Formato Dettagliato}}}
\author{}
\date{}



\begin{document}

\maketitle

\section*{\textcolor{sectioncolor}{Caso d'Uso 1: Creazione evento}}
\textcolor{textcolor}{
\textbf{Nome:} Creazione evento\\
\textbf{Portata:} Sistema di gestione eventi\\
\textbf{Livello:} Obiettivo utente\\
\textbf{Attore primario:} Organizzatore\\
\textbf{Parti interessate:} Organizzatore\\
\textbf{Precondizioni:}
\begin{itemize}
    \item L'organizzatore deve essere autenticato.
    \item Non deve esistere un altro evento creato nella stessa data.
\end{itemize}
\textbf{Garanzia di successo:} L'evento \`e stato creato con successo e salvato nel database.\\
\textbf{Scenario principale di successo:}
\begin{enumerate}
    \item L'organizzatore accede al sistema.
    \item Seleziona l'opzione "Crea evento".
    \item Inserisce i dettagli richiesti: data e luogo.
    \item Conferma la creazione dell'evento.
    \item Il sistema salva l'evento e invia una notifica di conferma.
\end{enumerate}
\textbf{Estensioni:}
\begin{itemize}
    \item Se i dettagli inseriti non sono validi, il sistema mostra un messaggio di errore specificando il problema.
    \item Se si tenta di creare un secondo evento nella stessa data, il sistema impedisce l'operazione e mostra un messaggio d'avviso.
\end{itemize}
\textbf{Requisiti speciali:} Il sistema deve garantire un'interfaccia user-friendly per l'inserimento dei dati.
}

\section*{\textcolor{sectioncolor}{Caso d'Uso 2: Aggiunta Staff}}
\textcolor{textcolor}{
\textbf{Nome:} Aggiunta Staff\\
\textbf{Portata:} Sistema di gestione eventi\\
\textbf{Livello:} Obiettivo utente\\
\textbf{Attore primario:} Organizzatore\\
\textbf{Parti interessate:} Organizzatore, Staff\\
\textbf{Precondizioni:} Deve esistere un evento attivo.\\
\textbf{Garanzia di successo:} Lo staff è aggiunto correttamente al database associato all'evento.\\
\textbf{Scenario principale di successo:}
\begin{enumerate}
    \item L'organizzatore seleziona un evento attivo.
    \item Accede alla sezione "Aggiungi staff".
    \item Inserisce i dati richiesti per ogni staff (nome, cognome,email...).
    \item Conferma l'operazione.
    \item Il sistema lo salva nel database.
\end{enumerate}
\textbf{Estensioni:}
\begin{itemize}
    \item Se i dati sono incompleti, il sistema mostra un messaggio d'errore.
\end{itemize}
}


\section*{\textcolor{sectioncolor}{Caso d'Uso 3: Apertura/chiusura vendite biglietti}}
\textcolor{textcolor}{
\textbf{Nome:} Apertura/chiusura vendite biglietti\\
\textbf{Portata:} Sistema di gestione eventi\\
\textbf{Livello:} Obiettivo utente\\
\textbf{Attore primario:} Organizzatore\\
\textbf{Parti interessate:} Organizzatore, Staff\\
\textbf{Precondizioni:} Deve esistere un evento attivo.\\
\textbf{Garanzia di successo:} Lo stato delle vendite viene aggiornato correttamente nel database.\\
\textbf{Scenario principale di successo:}
\begin{enumerate}
    \item L'organizzatore accede al sistema.
    \item Seleziona l'evento desiderato.
    \item Modifica lo stato delle vendite (aperto o chiuso).
    \item Il sistema aggiorna lo stato nel database e invia una notifica.
\end{enumerate}
\textbf{Estensioni:}
\begin{itemize}
    \item Se mancano informazioni essenziali sull'evento, il sistema mostra un messaggio d'errore.
\end{itemize}
\textbf{Requisiti speciali:} Il sistema deve chiudere automaticamente le vendite due settimane prima della data dell'evento.
}
\section*{\textcolor{sectioncolor}{Caso d'Uso 4: Aggiunta biglietti}}
\textcolor{textcolor}{
\textbf{Nome:} Aggiunta biglietti\\
\textbf{Portata:} Sistema di gestione eventi\\
\textbf{Livello:} Obiettivo utente\\
\textbf{Attore primario:} Organizzatore, staff\\
\textbf{Parti interessate:} Organizzatore, Staff\\
\textbf{Precondizioni:} Deve esistere un evento attivo.\\
\textbf{Garanzia di successo:} I biglietti sono stati aggiunti al database associato all'evento.\\
\textbf{Scenario principale di successo:}
\begin{enumerate}
    \item L'organizzatore seleziona un evento attivo.
    \item Accede alla sezione "Aggiungi Biglietto".
    \item Inserisce i dati richiesti per ogni biglietto (nome, cognome).
    \item Conferma l'operazione.
    \item Il sistema genera un codice identificativo univoco per ciascun biglietto e lo salva nel database.
\end{enumerate}
\textbf{Estensioni:}
\begin{itemize}
    \item Se i dati sono incompleti, il sistema mostra un messaggio d'errore.
    \item Se le vendite per l'evento sono chiuse, il sistema blocca l'operazione.
\end{itemize}
\textbf{Requisiti speciali:} Ogni biglietto deve avere un QR code generato automaticamente e univoco.


\section*{\textcolor{sectioncolor}{Caso d'Uso 4: Acquisto biglietto}}
\textcolor{textcolor}{
\textbf{Nome:} Acquisto biglietto\\
\textbf{Portata:} Sistema di gestione eventi\\
\textbf{Livello:} Obiettivo utente\\
\textbf{Attore primario:} Cliente\\
\textbf{Parti interessate:} Cliente, Organizzatore\\
\textbf{Precondizioni:} Le vendite per l'evento devono essere aperte.\\
\textbf{Garanzia di successo:} Il cliente riceve un biglietto valido con QR code associato.\\
\textbf{Scenario principale di successo:}
\begin{enumerate}
    \item Il sistema verifica il pagamento, genera il QR code e lo invia al cliente via email.
\end{enumerate}
\textbf{Estensioni:}
\begin{itemize}
    \item  Se il pagamento fallisce, il sistema annulla l'operazione e avvisa il cliente.
    \item  L'organizzatore può aggiungere, modificare ed eliminare un biglietto.
\end{itemize}
\textbf{Requisiti speciali:} Il sistema deve garantire pagamenti sicuri e generare QR code unici per ogni biglietto.
}

\section*{\textcolor{sectioncolor}{Caso d'Uso 5: Verifica accesso evento}}
\textcolor{textcolor}{
\textbf{Nome:} Verifica accesso evento\\
\textbf{Portata:} Sistema di gestione eventi\\
\textbf{Livello:} Obiettivo utente\\
\textbf{Attore primario:} Staff ingresso\\
\textbf{Parti interessate:} Cliente, Organizzatore\\
\textbf{Precondizioni:}
\begin{itemize}
    \item L'evento deve essere attivo.
    \item I dati dei partecipanti devono essere già registrati nel database.
\end{itemize}
\textbf{Garanzia di successo:} L'accesso del cliente è registrato come "presente" e il QR code viene invalidato.\\
\textbf{Scenario principale di successo:}
\begin{enumerate}
    \item Il cliente si presenta all'ingresso con il biglietto (QR code).
    \item Lo staff scansiona il QR code con il dispositivo.
    \item Il sistema verifica la validità del QR code.
    \item Se valido, registra la presenza nel database e invalida il QR code.
    \item Il sistema notifica l'accesso con successo allo staff.
\end{enumerate}
\textbf{Estensioni:}
\begin{itemize}
    \item Se il QR code non è valido (già utilizzato o inesistente), il sistema notifica l'errore.
\end{itemize}
\textbf{Requisiti speciali:} La verifica deve essere completata in meno di 2 secondi per gestire un alto volume di ingressi.
}

\section*{\textcolor{sectioncolor}{Caso d'Uso 6: Gestione guardaroba}}
\textcolor{textcolor}{
\textbf{Nome:} Gestione guardaroba\\
\textbf{Portata:} Sistema di gestione guardaroba\\
\textbf{Livello:} Sottofunzione\\
\textbf{Attore primario:} Staff guardaroba\\
\textbf{Parti interessate:} Cliente, Organizzatore\\
\textbf{Precondizioni:}
\begin{itemize}
    \item Il QR code del cliente deve essere valido.
    \item Il database deve essere accessibile per la registrazione.
\end{itemize}
\textbf{Garanzia di successo:} Il QR code viene registrato con un numero di gruppo incrementale e visualizzato allo staff.\\
\textbf{Scenario principale di successo:}
\begin{enumerate}
    \item Il cliente si presenta al guardaroba con il biglietto (QR code).
    \item Lo staff scansiona il QR code.
    \item Il sistema verifica la validità del QR code.
    \item Se valido, assegna un numero di gruppo incrementale.
    \item Registra il numero nel campo "guardaroba" del database.
    \item Mostra il numero allo staff per l'etichettatura degli oggetti del cliente.
\end{enumerate}
\textbf{Estensioni:}
\begin{itemize}
    \item  Se il QR code non è valido, il sistema notifica un errore e non registra l'operazione.
\end{itemize}
\textbf{Requisiti speciali:} L'assegnazione del numero di gruppo deve essere unica e incrementale.
}

\section*{\textcolor{sectioncolor}{Caso d'Uso 7: Monitoraggio scorte}}
\textcolor{textcolor}{
\textbf{Nome:} Monitoraggio scorte\\
\textbf{Portata:} Sistema di gestione scorte\\
\textbf{Livello:} Sottofunzione\\
\textbf{Attore primario:} Gestore scorte\\
\textbf{Parti interessate:} Organizzatore, Camerieri\\
\textbf{Precondizioni:}
\begin{itemize}
    \item Il database delle scorte deve essere inizializzato e aggiornato.
\end{itemize}
\textbf{Garanzia di successo:} Lo stato delle scorte viene visualizzato o aggiornato con successo.\\
\textbf{Scenario principale di successo:}
\begin{enumerate}
    \item Il gestore accede al modulo di gestione delle scorte.
    \item Visualizza le quantità attuali di cibo, ghiaccio e alcol.
    \item Se necessario, aggiorna le quantità disponibili.
    \item Il sistema salva i nuovi dati nel database.
\end{enumerate}
\textbf{Estensioni:}
\begin{itemize}
    \item  Se i dati inseriti non sono validi (valori negativi, formati errati), il sistema notifica un errore.
\end{itemize}
\textbf{Requisiti speciali:} Il monitoraggio deve essere in tempo reale per fornire dati aggiornati.
}

\section*{\textcolor{sectioncolor}{Caso d'Uso 8: Visualizzazione timeline servizio}}
\textcolor{textcolor}{
\textbf{Nome:} Visualizzazione timeline servizio\\
\textbf{Portata:} Sistema di gestione eventi\\
\textbf{Livello:} Obiettivo utente\\
\textbf{Attore primario:} Cameriere\\
\textbf{Parti interessate:} Organizzatore\\
\textbf{Precondizioni:}
\begin{itemize}
    \item Gli orari del servizio devono essere configurati dall'organizzatore.
    \item Le scorte devono essere registrate nel sistema.
\end{itemize}
\textbf{Garanzia di successo:} La timeline viene visualizzata correttamente sul dispositivo del cameriere.\\
\textbf{Scenario principale di successo:}
\begin{enumerate}
    \item Il cameriere accede all'interfaccia dedicata.
    \item Visualizza la timeline del servizio con gli orari preimpostati.
    \item Controlla le attività previste e i dati relativi alle scorte disponibili.
    \item Segue la timeline per il servizio agli ospiti.
\end{enumerate}
\textbf{Estensioni:}
\begin{itemize}
    \item  Se mancano dati (es. scorte non disponibili), il sistema notifica il problema al cameriere.
    \item  Se l'evento è stato annullato o non è attivo, il sistema blocca l'accesso alla timeline.
\end{itemize}
\textbf{Requisiti speciali:} L'interfaccia deve essere ottimizzata per dispositivi mobili e visualizzabile in tempo reale.
}

\end{document}
