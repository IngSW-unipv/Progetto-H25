\documentclass[a4paper,12pt]{article}
\usepackage[utf8]{inputenc}
\usepackage[italian]{babel}
\usepackage{hyperref}
\usepackage{amsmath}
\usepackage{xcolor}
\usepackage{geometry}
\usepackage{setspace}
\usepackage{graphicx}
\usepackage{fancyhdr}
\geometry{a4paper, margin=1.5in}

% Definizione di colori personalizzati
\definecolor{titlecolor}{RGB}{13, 82, 149}  % Blu scuro
\definecolor{sectioncolor}{RGB}{28, 69, 136} % Blu scuro pi\`u chiaro 
\definecolor{textcolor}{RGB}{0, 0, 0}    % Grigio scuro

\setstretch{1.5}

% Personalizzazione dell'intestazione
\pagestyle{fancy}
\fancyhf{} % Cancella intestazioni e pi\`e di pagina predefiniti
\fancyhead[L]{} % Lascia vuota la parte sinistra dell'intestazione
\fancyhead[R]{\includegraphics[width=1cm]{LogoAGA.jpeg}} % Logo in alto a destra
\fancyfoot[R]{\thepage} % Aggiunge il numero di pagina al centro del pi\`e di pagina

% Personalizzazione dello stile del titolo e delle sezioni
\title{\textcolor{titlecolor}{\Huge Requisiti Funzionali \vspace{0.2cm}}}
\author{}
\date{}

% Impostazione della dimensione del font
\renewcommand{\normalsize}{\fontsize{13pt}{16pt}\selectfont}
\renewcommand{\large}{\fontsize{15pt}{18pt}\selectfont}
\renewcommand{\Large}{\fontsize{17pt}{20pt}\selectfont}
\renewcommand{\LARGE}{\fontsize{20pt}{24pt}\selectfont}
\renewcommand{\Huge}{\fontsize{24pt}{28pt}\selectfont}

\begin{document}

\maketitle

\section*{\textcolor{sectioncolor}{Creazione Evento}}
\textcolor{textcolor}{
\begin{itemize}
    \item Il sistema deve permettere all’organizzatore di creare un nuovo evento.
    \item Il sistema deve permettere l'inserimento di dettagli come data e luogo dell'evento.
    \item Il sistema deve verificare la validità dei dati inseriti (data e luogo) e assicurarsi che non ci siano eventi programmati nella stessa data.
    \item Se i dati sono validi, il sistema deve salvare l'evento nel database e renderlo disponibile per la gestione dei biglietti.
    \item Il sistema deve notificare l’organizzatore in caso di errori nei dati inseriti.
    \item L’organizzatore aggiunge la lista di invitanti che hanno accesso alla pagina aggiunta biglietti.
\end{itemize}
}

\section*{\textcolor{sectioncolor}{Aggiunta Biglietti}}
\textcolor{textcolor}{
\begin{itemize}
    \item Il sistema deve permettere all’organizzatore e agli invitanti di aggiungere biglietti per un evento le cui vendite sono aperte.
    \item Il sistema deve permettere l'inserimento di dettagli per ogni biglietto, come nome, cognome, email.
    \item Il sistema deve verificare che le vendite per l'evento siano aperte prima di consentire l'aggiunta dei biglietti.
    \item Il sistema deve generare un codice identificativo univoco per ciascun biglietto, salvarlo nel database.
    \item Il sistema deve impedire ad un invitante di aggiungere più del numero di biglietti consentito.
    \item Il sistema deve impedire l'aggiunta di biglietti se le vendite per l'evento sono chiuse e informare l’organizzatore.
\end{itemize}
}

\section*{\textcolor{sectioncolor}{Apertura/Chiusura Vendite Biglietti}}
\textcolor{textcolor}{
\begin{itemize}
    \item Il sistema deve permettere all’organizzatore di modificare lo stato delle vendite (aperto/chiuso) per ogni evento attivo.
    \item Il sistema deve aggiornare il database con il nuovo stato delle vendite.
    \end{itemize}


\section*{\textcolor{sectioncolor}{Verifica Accesso Evento}}
\textcolor{textcolor}{
\begin{itemize}
    \item Il sistema deve permettere allo staff all’ingresso di scansionare il QR code dei clienti.
    \item Il sistema deve verificare la validità del QR code.
    \item Se il QR code è valido, il sistema deve registrare la presenza del cliente nel database.
    \item Il sistema deve invalidare il QR code per evitare ingressi multipli.
    \item Se il QR code non è valido, il sistema deve avvisare lo staff con un messaggio.
\end{itemize}
}

\section*{\textcolor{sectioncolor}{Gestione Guardaroba}}
\textcolor{textcolor}{
\begin{itemize}
    \item Il sistema deve permettere allo staff del guardaroba di scansionare il QR code dei clienti.
    \item Il sistema deve verificare la validità del QR code.
    \item Se il QR code è valido, il sistema deve assegnare un numero unico al cliente e registrarlo nel database.
    \item Il sistema deve mostrare il numero assegnato allo staff del guardaroba.
    \item Se il QR code non è valido, il sistema deve inviare un messaggio di errore allo staff.
    \item Il sistema deve generare numeri di guardaroba in ordine incrementale.
    \item Il sistema deve considerare un numero di grucce limitato.
\end{itemize}
}

\section*{\textcolor{sectioncolor}{Monitoraggio Scorte}}
\textcolor{textcolor}{
\begin{itemize}
    \item Il sistema deve permettere al gestore di visualizzare le scorte in tempo reale, come cibo, ghiaccio e alcol.
    \item Il sistema deve permettere al gestore di aggiornare le quantità delle scorte registrate.
    \item Il sistema deve segnalare la mancanza di dati relativi alle scorte con un messaggio di errore.
    \item Il sistema deve aggiornare le informazioni delle scorte in tempo reale per garantire accuratezza.
\end{itemize}
}

\section*{\textcolor{sectioncolor}{Visualizzazione Timeline Servizio}}
\textcolor{textcolor}{
\begin{itemize}
    \item Il sistema deve permettere ai camerieri di visualizzare una timeline chiara e leggibile delle attività e degli orari del servizio per l'evento attivo.
    \item Il sistema deve avvisare il cameriere in caso di mancanza di dati essenziali o disponibilità insufficiente delle scorte.
\end{itemize}
}

\end{document}
